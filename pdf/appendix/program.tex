
\section{Async Rattus Haskell example port}\label{appendix:program}

This is a Rizzo port of the Async Rattus example program from Fig. 4 in \citetitle{gebser_asynchronous_2023}.
We have adapted the program to use Rizzo's API and OCaml syntax, and added a port output for the console input as well.

% code snippet example
\begin{lstlisting}
open! Rizzo.Types
open! Rizzo.Signal
open! Rizzo.Channel

let _paper_example =
  let debug = false in
  let every_second, every_second_stop = clock_signal 1.0 in
  let port_input = mkSig_of_channel @@ port_input 9000 in
  let read_int = int_of_string_opt in
  let nats init = scan (fun n _ -> n + 1) init every_second in

  let console  = console_input () |> wait |> mkSig in
  let quit_sig = filter ((=) "quit") console in
  let show_sig = filter ((=) "show") console in
  let neg_sig  = filter ((=) "negate") console in
  let num_sig  = filter_map read_int console in
  let interleaved =
    interleave
      Fun.compose
      (map (fun _ n -> -n   ) ("" @: neg_sig))
      (map (fun m n -> m + n) (0  @: num_sig))
  in
  let rec nats' init =
    switchS (nats init) ((fun s n -> nats' (head s n)) <<| tail interleaved)
  in
  let nats_prim = nats' 0 in
  if debug then
   console_output (map (fun n -> "tick: " ^ string_of_int n) nats_prim);
  let show_nat = triggerL (fun _ n -> n) show_sig (nats_prim) in
  port_outputL Unix.inet_addr_loopback 9000 (mapL string_of_int show_nat);
  console_outputL (mapL (fun s -> "Number is: " ^ s) port_input);
  set_quit quit_sig;
  start_event_loop ();
  every_second_stop ()
\end{lstlisting}

\section{simple example.ml}\label{appendix:simple_example_program}

\begin{lstlisting}
open Rizzo.Types
open Rizzo.Signal
open Rizzo.Channel
let () =
    let console_channel = console_input () in
    let port_channel = port_input 9000 in

    let console_out_signal = mkSig_of_channel console_channel in
    let port_out_signal = mkSig_of_channel port_channel in

    console_outputL (mapL (fun s -> "From console: " ^ s) console_out_signal);
    console_outputL (mapL (fun s -> "From port: " ^ s) port_out_signal);

    port_outputL Unix.inet_addr_loopback 9000
    console_out_signal;
    start_event_loop ();
\end{lstlisting}

\section{Extended example.ml}\label{appendix:extended_example_program}

The program will take any input from the console and print it directly to the console prefixed with `From console: '. It will also send the same input to port 9000 on localhost, and print any input received on that port to the console prefixed with `From port: '.
It will then use filter to make a new signal that only contains the string `time' from the console input.
This signal will then be sampled with a clock signal that ticks every second, producing a new signal that contains the time in seconds since the program started whenever the `time' command is received from the console.

\begin{lstlisting}
open Rizzo.Types
open Rizzo.Signal
open Rizzo.Channel

let () =
    (* Setup input channels, signals *)
    let console_channel = console_input () in
    let port_channel = port_input 9000 in

    let console_in = mkSig_of_channel console_channel in
    let port_in = mkSig_of_channel port_channel in

    (* Write input to console out *)
    console_outputL (mapL (fun s -> "From console: " ^ s) console_in);
    console_outputL (mapL (fun s -> "From port: " ^ s) port_in);

    (* Send console input to port output *)
    port_outputL Unix.inet_addr_loopback 9000 console_in;

    (* Create clock signal to sample time every second *)
    let every_second, every_second_stop = clock_signal 1.0 in
    let start_time = head every_second in

    (* Signal that only updates on "time" commands *)
    let time_filter = filterL (fun s -> s = "time") console_in in
    
    (* Sample the clock when time command has been registered *)
    let sampled_console = sampleL time_filter every_second in

    (* Output the time to the console *)
    let formatted =
      (mapL (fun (_, f) -> string_of_float (f -. start_time))
          sampled_console) in
    console_outputL formatted;

    (* Start the event loop *)
    start_event_loop ();
    (* Stop the clock when the event loop is over *)
    every_second_stop ()
\end{lstlisting}
