
% The black art of writing fast code: some steps to improve optimization do not
% provide measurable benefits and are ineffective, but others enable the compiler
% to do further optimizations. Often, a good approach involves a combination of
% experimentation and analysis repeated ad infinitum until we see the desired
% results. In this paper we present things you should generally keep in mind when
% delving into performance optimization.

\section{Motivation}\label{motivation}

% something about why this is important and why we chose this topic

\subsubsection{Research question(s)}\label{research-questions}

%% some research questions we want to answer with this project
\begin{enumerate}
  \item How do we condense knowledge of CPU architecture and its influence on
  performance in such a way that a software development student at the level of
  a bachelor's degree can understand it?
  \item How do we coherently present this knowledge while abstracting
  unnecessary details?
  \item Can we utilize simple code examples and performance benchmarks to convey
  the information?
\end{enumerate}

\subsubsection{Methodology}\label{methodology}

% something about how we will do this project

