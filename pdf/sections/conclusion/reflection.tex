
% could for example include:
% 1. how do our results relate to previous work?
% 2. discuss further implications showing new ways for research
% 3. anticipate possible counterarguments
% 4. state the significance of this paper (self-absorbed much)

\subsection{Reflection}\label{reflection}

% some text about what we wanted to achieve with this project

\subsubsection{Evaluation}

We believe we have achieved our goal of providing a brief yet informative
introduction to Rizzo. We have presented the core concepts of Rizzo, including its
type system and operational semantics, while demonstrated how to use our OCaml library with examples. We have also discussed the challenges we faced during the implementation by chosing OCaml and how we addressed them. For example we had to deal with OCaml's eager evaluation strategy, which required us to implement thunks to delay computations until the next time step.

\subsubsection{Comparison to other works}

Something about how our work compares to other works...

\subsubsection{Further research}

What have held us back and what could be done in the future...?

No clue where I should put these thoughts, so putting them here for now:
Should all input be given to the user as signals, so the user has less concepts to think about? The channels are instantly converted to signals anyway, so why not just give the user signals directly?

Should we make a way to expose all signals as a later, and that way the user only ever has to think about one type of signal? This would make the API simpler, and easier to learn.

How would the user be able to create new input channels? Should we provide an interface for creating new channels, or should the user have to implement their own channels from scratch? and is there a domain of inputs all input types, and if so is that the OS interface? What else than console, env, files, network and clocks would be useful?
