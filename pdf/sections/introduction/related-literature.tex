
\subsection{Related literature}\label{related-literature}

% some text about what other people have done in this area
Library-based implementations of FRP can be traced back to FRAN~\cite{ElliottHudak97:Fran}, the original Haskell FRP library. FRAN introduced a purely functional interface for describing \emph{events}, values that become known at some point in time, and \emph{behaviours} which we have refered to as signals. This work kick started the development of FRP as a programming paradigm. 
However, since the original FRAN suffered from space leak problems, a lot of work has gone into fixing these. 
Later FRP libraries and languages propose different designs to avoid these problems using techniques such as richer type systems, arrow-based approaches (a generalisation of monads), and carefully restricted interfaces~\cite{neel_krishnaswami_2013,rizzo_modal_2025,gebser_asynchronous_2023,FRP_NOW,yampa_hudak_2003,adjs,froc_frp,reactive_banana}. 
For example, FRPNOW!~\cite{FRP_NOW} proposes a monadic interface to avoid some sources of space leaks, while Yampa~\cite{yampa_hudak_2003} replaces first-class signals with \emph{signal transformers} that is composed using arrows, creating a programming style that helps avoid space leaks. 

Other approaches address space leaks through modal types, similar to Rizzo~\cite{rizzo_modal_2025}. 
Examples include AdJS~\cite{adjs}, an FRP language which compiles to JavaScript based on the synchronous calculus of Krishnaswami~\cite{neel_krishnaswami_2013}, and Async Rattus~\cite{gebser_asynchronous_2023} which, like Rizzo, is asynchronous but is embedded in Haskell by a compiler plugin and a library for the language primitives.
% \todo{Mention `Rhine: FRP with Type-Level Clocks'?}

% A different line of work uses typed clocks to structure reactivity, e.g.
% Rhine, which tracks rates and clock domains at the type level. This allows
% programs to safely combine signals sampled at different rates and from different 
% clock domains~\cite{rhine}. While this approach can help avoid certain classes of space leaks, it does not directly address the core challenges that Rizzo targets, such as eliminating the need for a stable modality.
